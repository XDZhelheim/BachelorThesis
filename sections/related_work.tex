
\section{Related Work}
\textbf{Public Traffic Datasets.} There are several public traffic datasets which are frequently used for traffic prediction. They can be briefly categorized into three classes by spatial domain, which are \textbf{grid-based}, \textbf{sensor-based} and \textbf{road-network-based}.

For grid-based datasets, there are \textit{TaxiBJ}\cite{taxibj} that consists of the taxi in and out flow data in Beijing, and \textit{TaxiNYC} for taxis in New York City published by the New York City Taxi and Limousine Commission (TLC). For sensor-based datasets, \textit{METR-LA}\cite{DCRNN} and \textit{PEMS-BAY} are the most widely used datasets in urban traffic prediction area. In detail, \textit{PEMS-BAY} is collected from 325 sensors all over the San Jose bay area every 5 minutes. The traffic sensors can directly record the traffic flow of each road, which makes the dataset easy to handle and process. And for road-network-based datasets, \textit{Didi GAIA}'s open data has a good quality but they are seldom applied to build a model. It is GPS data containing taxi locations with timestamp that collected by \textit{Didi} company's mobile app, which is similar to ours. In conclusion, as suggested by Jiang and Luo\cite{surveyGNN}, most traffic prediction models are built upon traffic sensor datasets, while road-network-based datasets are mainly used for test. Therefore, we need to make better use of it.

\vspace{\baselineskip}

\textbf{Road Network Modeling.} Road network is the basic component of urban traffic system. To make use of the spatial information inside it, many approaches have been proposed. Statistical models are used to represent road network. For recent traffic prediction articles, Li et al.\cite{AAAI21} model road transition as a Markov Process over road network and use a first order Markov matrix to represent it. The growth of deep learning models makes it possible to model more complex road network and learn road characteristics efficiently. In basic GCN\cite{GCN0}, the authors use adjacency matrix to calculate Graph Laplacian Matrix in order to represent the whole graph. Lately, Wu et al.\cite{roadrep} proposed a hierarchical graph neural networks to capture both structural and functional characteristics of road network through several pre-defined attributes of each road. Wu et al.\cite{GWNET} use graph convolution to learn a new adjacency matrix of sensor graph, which is quite related to our work. To conclude, the two methods mentioned above need prior knowledge or history traffic state of roads. In contrast, our work is to learn a representation of each road to model its spatial characteristics only by trajectories.

\vspace{\baselineskip}

\textbf{Traffic Prediction Models.} Early attempts use traditional time series forecasting model including ARIMA\cite{ARIMA_pred} and VAR\cite{var_pred}, as well as machine learning techniques like k-NN\cite{knn_pred} and SVM\cite{svm_pred}. As mentioned in section 1, these models cannot capture the spatiotemporal dependency well. Many state-of-the-art deep neural networks have been proposed in the last several years. Yu et al.\cite{STGCN} proposed two different convolution blocks to capture spatial and temporal dependencies separately. Li et al.\cite{DCRNN} take advantage of seq2seq\cite{seq2seq} architecture and perform diffusion convolution on the graph. From our observation, few existing work leverage trajectories in traffic prediction. Hui et al.\cite{trajnet} extract the temporal features of roads by convolution with recent, daily-periodic and weekly-periodic traffic state data. Then they perform feature smoothing by propagating features through trajectories. On the contrary, our work attempts to combine the spatial representation that learned from trajectories into traffic state prediction models.
