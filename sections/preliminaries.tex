
\section{Preliminaries}
In this section, we will introduce the notations used in this paper and problem definitions in our task.

\subsection{Notations}
\begin{table}[htb]
    \begin{center}
        \caption{Notations}
        \label{notation_table}
        \begin{tabular}{cl}
            \toprule

            \textbf{Notation} & \textbf{Definition}                       \\

            \midrule

            $n_r$             & \#roads                                   \\
            $\mathcal R$      & road set                                  \\
            $r$               & a single road in $\mathcal R$             \\
            $\mathcal E$      & edge set                                  \\
            $A$               & adjacency matrix                          \\
            $\mathcal G$      & road network graph                        \\
            $\mathcal T$      & trajectory set                            \\
            $T$               & a trajectory in $\mathcal T$              \\
            $ts$              & timestamp                                 \\
            $s$               & speed                                     \\
            $E$               & road embedding matrix                     \\
            $\mathbf{e}_i$    & embedding vector for road $r_i$           \\
            $d_r$             & dimension of embedding vectors            \\
            $C$               & road correlation matrix                   \\
            $t$               & time interval                             \\
            $n_t$             & \#time intervals                          \\
            $X$               & traffic state matrix                      \\
            $\mathbf x_t$     & traffic state vector at time interval $t$ \\

            \bottomrule
        \end{tabular}
    \end{center}
\end{table}
The above table \ref{notation_table} gives the notations and their definitions.

\subsection{Problem Definition}
This section gives the definitions\cite{AAAI21} of the concepts and tasks occurred in this paper.
\begin{definition}[Road Network Graph]
    The road network can be represented by a directed graph $\mathcal{G}=(\mathcal{R}, \mathcal{E}, A)$, where $\mathcal{R}=\{r_1, r_2, \dots, r_{n_r}\}$ is a finite set of roads that each $r_i$ stands for a real road in the road network. $\mathcal{E}$ is the set of directed edges where $(r_i, r_j)\in \mathcal{E}$ indicates that there is a directed edge from $r_i$ to $r_j$, i.e. $r_j$ is the downstream road in the road network. $A \in [0, 1]^{n_r\times n_r}$ is the adjacency matrix whose entry $A_{ij}$ is a binary value that indicates whether there exists an edge $(r_i, r_j)\in\mathcal{E}$.
\end{definition}

\begin{definition}[Trajectory]
    Given a road network graph $\mathcal{G}=(\mathcal{R}, \mathcal{E}, A)$, a trajectory $T=[(r_1, s_1, ts_1), (r_2, s_2, ts_2), \dots, (r_l, s_l, ts_l)]$ is a sequence of (road, speed, timestamp) tuples. Each tuple $(r_i, s_i, ts_i)$ specifies that the vehicle is driving on $r_i$ with speed $s_i$ at timestamp $ts_i$. Besides, $\forall i=1, 2, \dots, l-1$, $r_i\neq r_{i+1}$ and $(r_i, r_{i+1})\in\mathcal{E}$.
\end{definition}

\begin{definition}[Traffic State]
    Traffic state stands for the traffic flow or speed of a road during a particular time interval. Traffic flow is defined as the number of vehicles passing by the road, and traffic speed is the average speed of these vehicles. For a road graph $\mathcal{G}=(\mathcal{R}, \mathcal{E}, A)$, we use $X\in\RR^{n_r\times n_t}$ to record the traffic state of each time interval. For time interval $t$, $\mathbf{x}_t=X_{:, t}\in\RR^{n_r}$ represents the traffic state of all roads during $t$.
\end{definition}

\begin{problem}[Road Correlation]
    Given a road network graph $\mathcal{G}=(\mathcal{R}, \mathcal{E}, A)$, find a road correlation function $Cor$ which takes two roads as input and returns a real number $0\leqslant Cor(r_i, r_j)\leqslant 1$ to quantify the spatial dependency between two roads $r_i$ and $r_j$. The value is bigger if the two roads have a stronger dependency, e.g. $r_i$ is the only way to $r_j$. The road correlation matrix $C$ stores all the correlation values s.t. $C_{ij}=Cor(r_i, r_j)$.
\end{problem}

\begin{problem}[Traffic State Prediction]
    Given a road network graph $\mathcal{G}=(\mathcal{R}, \mathcal{E}, A)$, find a function $f$ and its parameter set $\Theta$ s.t. given historical traffic states $\{\mathbf{x}_{t-\tau_{in}+1}, \mathbf{x}_{t-\tau_{in}}, \dots, \mathbf{x}_t \}$ for an input window $\tau_{in}$, $f$ estimates the most likely traffic states $\{\mathbf{x}_{t+1}, \mathbf{x}_{t+2}, \dots, \mathbf{x}_{t+\tau_{out}} \}$ for an output window $\tau_{out}$.
    \begin{equation}
        \hat X_{:, t+1:t+\tau_{out}}=f_\Theta(X_{:, t-\tau_{in}+1:t-1})=\mathop{\arg\max}_{X_{:, t+1:t+\tau_{out}}} p(X_{:, t+1:t+\tau_{out}}|X_{:, t-\tau_{in}+1:t-1})
    \end{equation}
\end{problem}

\subsection{Overview}
