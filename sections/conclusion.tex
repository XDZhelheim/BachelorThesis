
\section{Conclusion and Future Work}
In this paper, we first investigated on the existing traffic prediction works. The shortage of them is that the spatial dependency is expressed only by the static relationship among roads, specifically, the adjacency matrix $A$. Therefore, we proposed the concept of \textit{trajectory-based road correlation} which extracts dynamic spatial dependency through vehicle trajectories. Then we illustrated the construction of our trajectory dataset from raw GPS data, including data cleaning, road network simplification, map matching and data interpolation, resulting in the traffic state matrix $X$. After that, the trajectory next-hop model is built to learn road embeddings based on a LSTM neural network. The \textit{road correlation matrix} $C$ is calculated by the embedding matrix $E$ in the model. Based on that, the $k$-NN approach for the computing of \textit{refined adjacency matrix} $A'$ is given. Then it is integrated into SOTA traffic prediction models to improve their performance. Extensive experiment results proved the effectiveness and robustness of our ideas and proposed methods. Following, a case study demonstrated the rationality of the road correlation.

However, the \textit{road correlation matrix} $C$ is still not dynamic enough, since it is learned by the trajectories over the whole time range. To strengthen its temporality, we are planning to extend our current work to study time-varying road correlations by splitting $C$ into each hour and build a time-dynamic traffic prediction model. In this case, we expect that the $C$-version can have better performance than baselines. In addition, the accuracy of the next-hop prediction model is kind of low. Therefore, it is worth trying to utilize more modern sequence prediction models for the learning of road embeddings. For the dataset construction, there is still a lot that can be optimized, such as the accuracy of data interpolation which can be improved by leveraging other interpolation algorithms rather than linear. Finally, for the experiments, it is necessary to involve more baseline models to prove the extensive effectiveness of the proposed road correlation.
